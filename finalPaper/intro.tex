
Behavior Trees are structures generally used to dictating a robots behavior while acomplishing a complicated task. While they provide an easy and understandable structure, they provide no safety guaruntees for the robots behavior, and it is often difficult to verify that an undesirable behavior will not occur. Within this paper, we atempt to remedy this by converting behavior trees to state machines. This allows us to mark certain states as unsafe, especially in product spaces generated by combining two robots behaviors. By doing so we can check reachability to these unsafe states, and determine how these states were reached within the Behavior Trees.


\subsection{Behavior Trees}
In robotics, Behavior Trees are used to break down a desired task into subsequent subtasks using two types of control nodes: Sequence nodes and Selector nodes. 

\subsection{Related Works}