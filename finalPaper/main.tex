\documentclass[twoside,11pt]{report}

% Any additional packages needed should be included after AMP-report-style.
% Note that AMP-report-style.sty includes epsfig, amssymb, natbib and graphicx,
% and defines many common macros, such as 'proof' and 'example'.
%
% It also sets the bibliographystyle to plainnat; for more information on
% natbib citation styles, see the natbib documentation.

% Available options for package AMP-report-style are:
%
%   - abbrvbib : use abbrvnat for the bibliography style
%   - nohyperref : do not load the hyperref package
%   - preprint : remove JMLR specific information from the template,
%         useful for example for posting to preprint servers.
%
% Example of using the package with custom options:
%

\usepackage{AMP-report-style}

% Definitions of handy macros can go here

\usepackage{lastpage}
\usepackage{xcolor}
\usepackage{graphicx}
\usepackage{float}
\usepackage{booktabs}
\newcommand{\comment}[1]{\textcolor{red}{#1}}

\jmlrheading{}{}{1-\pageref{LastPage}}{Dec. 5, 2025}{}{}{}

% Short headings should be running head and authors last names

% \ShortHeadings{}{One and Two}
\firstpageno{1}

\begin{document}

\title{Formal Safety Analysis of Multi-Robot Behavior Trees}

\author{\name Stefan Caldararu \email stefan.caldararu@colorado.edu\\
        \addr Department of Computer Science\\
        University of Colorado Boulder
        \AND
        \name Zack Allen \email zack.allen@colorado.edu\\ 
        \addr Department of Robotics\\
        University of Colorado Boulder      
}

\editor{}

\maketitle

\begin{abstract}%   <- trailing '%' for backward compatibility of .sty file
    Behavior Trees are structures commonly used in robotics breaking down complicated tasks into subtasks and organizing them into easily readable and programable structures. Unfortunately, it is difficult to perform a formal safety analysis on Behavior Trees, especially when having multiple robots performing different tasks in parallel. In this paper, we algorithmically convert Behavior Trees to basic State Machines, and proceed to combine two different constructed State Machines by taking the product space. This process allows for a simple safety check using reachability of the system to unsafe states. We apply this technique to a complicated Electric Vehicle battery dissasembly task, where a humanoid robot and 6DOF robotic arm are required to operate collaboratively. This demonstrates a specific application for this generalizable method.
\end{abstract}


%%%%%%%%%%%%%%%%%%%%%%%%%%%%%%%%%
\section{Introduction}

Behavior Trees are structures generally used to dictating a robots behavior while acomplishing a complicated task. While they provide an easy and understandable structure, they provide no safety guaruntees for the robots behavior, and it is often difficult to verify that an undesirable behavior will not occur. Within this paper, we atempt to remedy this by converting behavior trees to state machines. This allows us to mark certain states as unsafe, especially in product spaces generated by combining two robots behaviors. By doing so we can check reachability to these unsafe states, and determine how these states were reached within the Behavior Trees.


\subsection{Behavior Trees}
In robotics, Behavior Trees are used to break down a desired task into subsequent subtasks using two types of control nodes: Sequence nodes and Selector nodes. 

\subsection{Related Works}



%%%%%%%%%%%%%%%%%%%%%%%%%%%%%%%%%
\section{Problem Description}

We apply this proposed methodology to an Electric Vehicle (EV) battery dissasembly task. Within this environment, a H-1 Unitree Robot \comment{cite} and UR16e Robotic arm \comment{cite} are tasked with removing the protective cover from an EV battery. The first step to this process is removal of nuts fastening this protective cover. There are three main steps to this process: First, a scan of the battery must be complete to identify nut locations. Following this, the robotic arm will go from nut to nut, unscrewing each one. The humanoid will follow the robotic arm removing each nut once it has been unscrewed. We focus on the latter two steps, assuming nut locations are already known. This workspace is shown in Figure~\ref{fig:workspace}.

\begin{figure}[h!]
    \centering
    \includegraphics[width=0.5\textwidth]{humanoid_and_battery.png}
    \caption{An H1-Unitree and UR16e positioned to collaborate on removing nuts from an EV battery protective cover.}
    \label{fig:workspace}
\end{figure}

\subsection{Behavior Tree Formulation}
For both of the tasks at hand, we construct Behavior Trees to dictate each robots behavior. To ensure safety, each robot must not act within a nut workspace while the other robot is there. Behavior Trees for each robot are shown in Figure~\ref{fig:BT_humanoid} and Figure~\ref{fig:BT_arm}.

\begin{figure}[h!]
    \centering
    \includegraphics[width=0.5\textwidth]{BT_humanoid.png}
    \caption{An H1-Unitree and UR16e positioned to collaborate on removing nuts from an EV battery protective cover.}
    \label{fig:BT_humanoid}
\end{figure}

\begin{figure}[h!]
    \centering
    \includegraphics[width=0.5\textwidth]{BT_arm.png}
    \caption{An H1-Unitree and UR16e positioned to collaborate on removing nuts from an EV battery protective cover.}
    \label{fig:BT_arm}
\end{figure}

It can be seen that the robotic arm will perform the following checks as unscrewing each arm: First it will check if the bolt at the current location is already unscrewed. If so, it proceeds to the next bolt. Otherwise, it checks that the humanoid is not currently at the bolt, and ensures it is at the bolt location, and then proceeds to unscrew the bolt.

%%%%%%%%%%%%%%%%%%%%%%%%%%%%%%%%%
\section{Methodology} 




%%%%%%%%%%%%%%%%%%%%%%%%%%%%%%%%%
\section{Results} 


%%%%%%%%%%%%%%%%%%%%%%%%%%%%%%%%%
\section{Conclusion} 




%%%%%%%%%%%%%%%%%%%%%%%%%%%%%%%%%
\acks{}


%%%%%%%%%%%%%%%%%%%%%%%%%%%%%%%%%
% remove the following if Appendix is not needed.
\appendix
\section{}

\section{}


%%%%%%%%%%%%%%%%%%%%%%%%%%%%%%%%%
\vskip 0.2in
\bibliography{main.bib}

\end{document}
