\documentclass[twoside,11pt]{report}

% Any additional packages needed should be included after AMP-report-style.
% Note that AMP-report-style.sty includes epsfig, amssymb, natbib and graphicx,
% and defines many common macros, such as 'proof' and 'example'.
%
% It also sets the bibliographystyle to plainnat; for more information on
% natbib citation styles, see the natbib documentation.

% Available options for package AMP-report-style are:
%
%   - abbrvbib : use abbrvnat for the bibliography style
%   - nohyperref : do not load the hyperref package
%   - preprint : remove JMLR specific information from the template,
%         useful for example for posting to preprint servers.
%
% Example of using the package with custom options:
%

\usepackage{AMP-report-style}

% Definitions of handy macros can go here

\usepackage{lastpage}
\usepackage{xcolor}
\usepackage{graphicx}
\usepackage{float}
\usepackage{booktabs}
\newcommand{\comment}[1]{\textcolor{red}{#1}}

\jmlrheading{}{}{1-\pageref{LastPage}}{Dec. 5, 2025}{}{}{}

% Short headings should be running head and authors last names

% \ShortHeadings{}{One and Two}
\firstpageno{1}

\begin{document}

\title{Formal Safety Analysis of Multi-Robot Behavior Trees}

\author{\name Stefan Caldararu \email stefan.caldararu@colorado.edu\\
        \addr Department of Computer Science\\
        University of Colorado Boulder
        \AND
        \name Zack Allen \email zack.allen@colorado.edu\\ 
        \addr Department of Robotics\\
        University of Colorado Boulder      
}

\editor{}

\maketitle

\begin{abstract}%   <- trailing '%' for backward compatibility of .sty file
    Behavior Trees are structures commonly used in robotics breaking down complicated tasks into subtasks and organizing them into easily readable and programable structures. Unfortunately, it is difficult to perform a formal safety analysis on Behavior Trees, especially when having multiple robots performing different tasks in parallel. In this paper, we algorithmically convert Behavior Trees to basic State Machines, and proceed to combine two different constructed State Machines by taking the product space. This process allows for a simple safety check using reachability of the system to unsafe states. We apply this technique to a complicated Electric Vehicle battery dissasembly task, where a humanoid robot and 6DOF robotic arm are required to operate collaboratively. This demonstrates a specific application for this generalizable method.
\end{abstract}


%%%%%%%%%%%%%%%%%%%%%%%%%%%%%%%%%
\section{Introduction}

Behavior Trees are structures generally used to dictating a robots behavior while acomplishing a complicated task. While they provide an easy and understandable structure, they provide no safety guaruntees for the robots behavior, and it is often difficult to verify that an undesirable behavior will not occur. Within this paper, we atempt to remedy this by converting behavior trees to state machines. This allows us to mark certain states as unsafe, especially in product spaces generated by combining two robots behaviors. By doing so we can check reachability to these unsafe states, and determine how these states were reached within the Behavior Trees.


\subsection{Behavior Trees}
In robotics, Behavior Trees are used to break down a desired task into subsequent subtasks using two types of control nodes: Sequence nodes and Selector nodes. 

\subsection{Related Works}



%%%%%%%%%%%%%%%%%%%%%%%%%%%%%%%%%
\section{Problem Description}
We apply this proposed methodology to an Electric Vehicle (EV) battery dissasembly task. Within this environment, a Unitree H1-2 Robot and UR16e Robotic arm are tasked with removing the protective cover from an EV battery.

\begin{figure}[H]
    \centering
    \includegraphics[width=0.3\textwidth]{humanoid_and_battery.png}
    \caption{A Unitree H1-2 and UR16e positioned to collaborate on removing  bolts from an EV battery protective cover.}
    \label{fig:workspace}
\end{figure}

Figure~\ref{fig:workspace} shows the two robots positioned to collaborate on removing the EV battery protective cover. The first step in the removal process is a visual and depth scan of the protective cover to identify and localize the bolts. Following this, the robotic arm will go from bolt to bolt, unscrewing each one. The humanoid will follow the robotic arm removing each bolt once it has been unscrewed.

\subsection{Behavior Tree Formulation}
For both of the tasks at hand, we construct Behavior Trees to dictate each robots behavior. To ensure safety, each robot must not act within the same bolt workspace simultaneously. Behavior Trees for each robot are shown below in Figures ~\ref{fig:BT_humanoid} and ~\ref{fig:BT_arm}.

\begin{figure}[H]
    \centering
    \begin{subfigure}[t]{0.3\textwidth}
        \centering
        \includegraphics[width=\textwidth]{BT_arm.png}
        \caption{Behavior Tree for the UR16e robotic arm.}
        \label{fig:BT_arm}
    \end{subfigure}
    \hspace{0.04\textwidth} % Adjust this if needed
    \begin{subfigure}[t]{0.3\textwidth}
        \centering
        \includegraphics[width=\textwidth]{BT_humanoid.png}
        \caption{Behavior Tree for the Unitree H1-2 humanoid robot.}
        \label{fig:BT_humanoid}
    \end{subfigure}
    \caption{Behavior Trees for both robots.}
    \label{fig:behavior_trees}
\end{figure}

It can be seen that the robotic arm will perform the following checks as unscrewing each bolt: First it will check if the bolt at the current location is already unscrewed. If so, it proceeds to the next bolt. Otherwise, it checks that the humanoid is not currently at the bolt, and ensures it is at the bolt location, and then proceeds to unscrew the bolt. Similarly, the humanoid first checks if the bolt is already removed, and if not, checks that the robotic arm is not currently at the bolt, and then proceeds to remove the bolt.

%%%%%%%%%%%%%%%%%%%%%%%%%%%%%%%%%
\section{Methodology}
For formal safety analysis, we propose the following methodology: We Algorithmically construct State Machines 

\subsection{Algorithmic Construction of State Machines}

\subsection{Reduction of Product Space}



%%%%%%%%%%%%%%%%%%%%%%%%%%%%%%%%%
\section{Results} 
Through this safety analysis, we identified an error in our original Behavior Tree design. There was an unsafe state where the humanoid would attempt to unscrew a bolt that the robotic arm hadn't already unscrewed. Therefore, we were able to correct the Behavior Trees to ensure that the robotic arm always unscrews the bolt before the humanoid attempts to remove it. A partial example run demonstrating this failure of the system is shown in Figure~\ref{fig:bad}.

\begin{figure}[H]
        \centering
        \includegraphics[width=\textwidth]{BadRun.png}
        \caption{An example run demonstrating undesirable behavior.}
        \label{fig:bad}
\end{figure}

\begin{figure}[H]
    \centering
    \begin{subfigure}[b]{0.3\textwidth}
        \centering
        \includegraphics[width=\textwidth]{original_bt.png}
        \caption{Original Behavior Tree for the Unitree H1-2 humanoid robot.}
        \label{fig:BT_arm}
    \end{subfigure}
    \hspace{0.04\textwidth} % Adjust this if needed
    \begin{subfigure}[b]{0.3\textwidth}
        \centering
        \includegraphics[width=\textwidth]{fixed-bt.png}
        \caption{Fixed Behavior Tree for the Unitree H1-2 humanoid robot.}
        \label{fig:BT_humanoid}
    \end{subfigure}
    \caption{Behavior Trees for both robots.}
    \label{fig:correction_bt}
\end{figure}

Figure \ref{fig:correction_bt} demonstrates the effectiveness of our methodology in identifying and rectifying safety issues in multi-robot Behavior Trees. For this simple example, the process required only a single iteration to achieve a safe design. More complex scenarios may necessitate multiple iterations, but our approach provides a clear framework for systematically enhancing the safety of multi-robot systems through formal analysis and targeted BT refinement.

%%%%%%%%%%%%%%%%%%%%%%%%%%%%%%%%%
\section{Conclusion} 
This work demonstrates a practical and repeatable process for evaluating the safety of multi-robot systems controlled by Behavior Trees. By converting each robot’s BT into a formal state-machine representation, composing those machines into a unified product system, and performing unsafe-state reachability analysis, we gain the ability to both detect and trace hazardous interactions that might otherwise go unnoticed until testing. This approach also supports iterative improvement: once unsafe paths are identified, modifications to the original Behavior Trees can be performed and re-verified until the behavior is confirmed safe.

Looking ahead, we aim to extend the methodology in several key directions. First, we will conduct a broader multi-robot investigation by building complete disassembly Behavior Trees for both the UR16e arm and the humanoid robot, then analyzing the full product state machine for safety flaws and logical inconsistencies. We also seek to automate the human-dependent refinement steps, including algorithmic detection of physically impossible transitions and automated identification of unsafe configurations based on workspace or task specifications. Integrating agentic AI methods may enable automatic modification of Behavior Trees followed by immediate verification through state-machine conversion, creating a closed-loop design and validation system.

Finally, we are interested in exploring treelikeness metrics and potential methods for converting or abstracting the product state machine back into a behavior-tree-like structure. Such research could provide new insights into how multiple BTs interact when composed and may lead to principled techniques for generating safe collaborative behaviors directly from their combined execution space.



%%%%%%%%%%%%%%%%%%%%%%%%%%%%%%%%%
\acks{}


%%%%%%%%%%%%%%%%%%%%%%%%%%%%%%%%%
% remove the following if Appendix is not needed.
\appendix
\section{}

\section{}


%%%%%%%%%%%%%%%%%%%%%%%%%%%%%%%%%
\vskip 0.2in
\bibliography{main.bib}

\end{document}
