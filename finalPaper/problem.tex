We apply this proposed methodology to an Electric Vehicle (EV) battery dissasembly task. Within this environment, a Unitree H1-2 Robot and UR16e Robotic arm are tasked with removing the protective cover from an EV battery.

\begin{figure}[H]
    \centering
    \includegraphics[width=0.3\textwidth]{humanoid_and_battery.png}
    \caption{A Unitree H1-2 and UR16e positioned to collaborate on removing  bolts from an EV battery protective cover.}
    \label{fig:workspace}
\end{figure}

Figure~\ref{fig:workspace} shows the two robots positioned to collaborate on removing the EV battery protective cover. The first step in the removal process is a visual and depth scan of the protective cover to identify and localize the bolts. Following this, the robotic arm will go from bolt to bolt, unscrewing each one. The humanoid will follow the robotic arm removing each bolt once it has been unscrewed.

\subsection{Behavior Tree Formulation}
For both of the tasks at hand, we construct Behavior Trees to dictate each robots behavior. To ensure safety, each robot must not act within the same bolt workspace simultaneously. Behavior Trees for each robot are shown below in Figures ~\ref{fig:BT_humanoid} and ~\ref{fig:BT_arm}.

\begin{figure}[H]
    \centering
    \begin{subfigure}[t]{0.3\textwidth}
        \centering
        \includegraphics[width=\textwidth]{BT_arm.png}
        \caption{Behavior Tree for the UR16e robotic arm.}
        \label{fig:BT_arm}
    \end{subfigure}
    \hspace{0.04\textwidth} % Adjust this if needed
    \begin{subfigure}[t]{0.3\textwidth}
        \centering
        \includegraphics[width=\textwidth]{BT_humanoid.png}
        \caption{Behavior Tree for the Unitree H1-2 humanoid robot.}
        \label{fig:BT_humanoid}
    \end{subfigure}
    \caption{Behavior Trees for both robots.}
    \label{fig:behavior_trees}
\end{figure}

It can be seen that the robotic arm will perform the following checks as unscrewing each bolt: First it will check if the bolt at the current location is already unscrewed. If so, it proceeds to the next bolt. Otherwise, it checks that the humanoid is not currently at the bolt, and ensures it is at the bolt location, and then proceeds to unscrew the bolt. Similarly, the humanoid first checks if the bolt is already removed, and if not, checks that the robotic arm is not currently at the bolt, and then proceeds to remove the bolt.